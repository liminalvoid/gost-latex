% Настройки

% Импорт конфига
% Конфиг

% Кафедра
\newcommand{\department}{Название кафедры}

% Тип работы (практическая, лабораторная)
\newcommand{\worktype}{практической}

% Номер работы
\newcommand{\worknumber}{1}

% Номер варианта
\newcommand{\optionnumber}{1}

% Название работы (для перехода части названия
% на следующую строку необходимо добавить два "/")
\newcommand{\worktitle}{Cменить название работы}

% Дисциплина
\newcommand{\subject}{Сменить название дисциплины}

% Шифр группы (НН/о-ГГ-Г-о)
\newcommand{\grouptitle}{НН/о-ГГ-Г-о}

% Должность (ст. пр., доцент)
\newcommand{\jobtitle}{ст. пр.}

% Преподаватель
\newcommand{\teacher}{Петров П. П.}

% Фамилия автора с инициалами
\newcommand{\name}{Иванов И. И.}


% Если возникают проблемы при компиляции с
% данной строкой, необходимо установить на
% компьютер Times New Roman
\setmainfont{Times New Roman}

% Размер красной строки
\parindent=1.25cm

% Отступ между абзацами
\parskip=0pt

% Разрешить переносить слоги в 2 буквы
% (стандартное значение 3)
\righthyphenmin=2

% Полуторный межстрочный интервал
\linespread{1.3}

% Настройка заголовка оглавления
\addto\captionsrussian{\renewcommand{\contentsname}{Содержание}}

% Формат оглавления
\renewcommand{\cftaftertoctitle}{\thispagestyle{plain}}
\renewcommand\cfttoctitlefont{\hfill\fontsize{16pt}{16pt}\selectfont\bfseries\MakeUppercase}
\renewcommand\cftaftertoctitle{\hfill}

% Смена точек на нижние подчеркивания
\renewcommand{\cftdot}{\_}

% Удаление расстояний в содержании
\renewcommand{\cftdotsep}{0}
\cftsetpnumwidth{0.5em}
\setlength{\cftbeforesecskip}{0em}
\renewcommand{\cftsecaftersnum}{.}
\renewcommand{\cftsubsecaftersnum}{.}
\renewcommand{\cftsubsubsecaftersnum}{.}
\renewcommand{\cftsecnumwidth}{1em}
\renewcommand{\cftsubsecnumwidth}{1.7em}
\renewcommand{\cftsubsubsecnumwidth}{2.4em}

% Добавить точки у разделов в оглавлении
\renewcommand{\cftsecleader}{\cftdotfill{\cftdotsep}} 

% Формат секций и подсекций
\titleformat{\section}{\parskip=6pt\filcenter\fontsize{16pt}{16pt}\selectfont\bfseries\MakeUppercase}{\thesection}{.0em. }{}
\titleformat{\subsection}{\bfseries}{\thesubsection}{.0em. }{}
\titleformat{\subsubsection}{\bfseries}{\thesubsubsection}{.0em. }{}

% Начинать раздел с новой страницы
\AddToHook{cmd/section/before}{\clearpage}

\renewenvironment{abstract}{\clearpage\section*{\MakeUppercase{\abstractname}}}{\clearpage}

% Горизонтальный отступ у элемента списка
\labelwidth=1.25cm 

% Настройка положения нумерации страниц
\fancyhf{}
\renewcommand{\headrulewidth}{0pt}
\pagestyle{fancy}
\fancyhead[R]{\thepage}
\setlength{\headheight}{35pt}

% Ненумерованные списки разной вложенности
\renewcommand\labelitemi{--}
\renewcommand\labelitemii{--}
\renewcommand\labelitemiii{--}
\renewcommand\labelitemiv{--}

% Нумерованные списки разной вложенности
\renewcommand\labelenumi{\arabic{enumi})}
\renewcommand\labelenumii{\asbuk{enumii})}
\renewcommand\labelenumiii{\arabic{enumiii})}
\renewcommand\labelenumiv{\asbuk{enumiv})}

\makeatletter

% Буквы для нумерации списка (исключены ё, з, щ, ч, ъ, ы, ь)
% Подробнее https://ctan.math.illinois.edu/macros/latex/required/babel/contrib/russian/russianb.pdf 
\def\russian@alph#1{\ifcase#1\or
    а\or б\or в\or г\or д\or е\or ж\or
    и\or к\or л\or м\or н\or о\or п\or 
    р\or с\or т\or у\or ф\or х\or ц\or 
    ш\or э\or ю\or я\else\@ctrerr\fi}
\def\russian@Alph#1{\ifcase#1\or
    А\or Б\or В\or Г\or Д\or Е\or Ж\or
    И\or К\or Л\or М\or Н\or О\or П\or 
    Р\or С\or Т\or У\or Ф\or Х\or Ц\or 
    Ш\or Э\or Ю\or Я\else\@ctrerr\fi}

% Очистка страницы с содержанием
\fancypagestyle{plain}{
    % clear all header and footer fields
    \fancyhf{}
    % clear all header fields
    \fancyhead{}
    % clear all footer fields
    \fancyfoot{}
    \fancyhead[RO]{\thepage}
    \fancyhead[LE]{\thepage}
    \setlength{\headheight}{35pt}
    \renewcommand{\headrulewidth}{0pt}
    \renewcommand{\footrulewidth}{0pt}
}

% Разделы в оглавлении без выделения жирным,
% в верхнем регистре
\patchcmd{\l@section}{#1}{\textnormal{#1}}{}{}

% Страницы без выделения жирным
\patchcmd{\l@section}{#2}{\textnormal{#2}}{}{}

\apptocmd{\appendix}{
    \renewcommand{\thesection}{\Asbuk{section}}

    % Изменение формата раздела приложения
    \titleformat{\section}{\filcenter\fontsize{16pt}{16pt}\selectfont\bfseries}{}{0pt}{\MakeUppercase{\appendixname}~\thesection \\}{}{}
    
    \let\oldsec\section
    \renewcommand{\section}{
        \clearpage
        \phantomsection
        \refstepcounter{section}
        \addcontentsline{toc}{section}{\appendixname~\thesection}

        % Нумерация раздела после названия
        \oldsec*}
}
\makeatother

% Выравнивание по левому краю надписи таблицы
\captionsetup[table]{justification=raggedright, singlelinecheck=false}

% Переопределение caption из babel
\addto\captionsrussian{\renewcommand{\figurename}{Рисунок}}

% Метаданные PDF документа, отключение подсветки ссылок
\hypersetup{pdftitle={}, pdfauthor={\name}, colorlinks=false, pdfborder={0 0 0}}

% Настройка стилей листингов
\renewcommand{\lstlistingname}{Листинг}

\lstset{
    breaklines=true,
    keepspaces=true,
    showstringspaces=false
}

% Путь для изображений
\graphicspath{{sources/images}}
