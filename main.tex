% % !TeX program = xelatex
\documentclass[14pt, a4paper, titlepage]{extarticle}

%%%%%%%%%% Блок определения подключаемых пакетов %%%%%%%%%%

\usepackage[english,main=russian]{babel}
\usepackage{fontspec}
\usepackage{tabularx}
\usepackage{fancyhdr}
% Поменять гарнитуру в формулах на Times New Roman
\usepackage{newtxmath}
\usepackage[left=30mm, right=15mm, top=20mm, bottom=20mm]{geometry}
% Красная строка у первого абзаца раздела
\usepackage{indentfirst}
\usepackage{graphicx}
% Добавить раздел оглавление в оглавление
\usepackage{tocbibind}
% underline some lines
\usepackage[normalem]{ulem} 
\usepackage{tocloft}
% Настройка раздела, подраздела, подподраздела
\usepackage{titlesec}
% Настройка пунктуации
\usepackage[labelsep=endash]{caption}
% Пакет делает ссылки кликабельными и дает возможность добавить метаданные .pdf документа
\usepackage{hyperref}
% Для картиночек
\usepackage{float}
% Для кода
\usepackage{listings}

%%%%%%%%%% Блок настроек %%%%%%%%%%

% Если возникают проблемы при компиляции с данной строкой, необходимо установить на компьютер Times New Roman
\setmainfont{Times New Roman}

% Размер красной строки
\parindent=1.25cm

% Отступ между абзацами
\parskip=0pt

% Разрешить переносить слоги в 2 буквы (стандартное значение 3)
\righthyphenmin=2

% полуторный межстрочный интервал
\linespread{1.3}

% Настройка заголовка оглавления
\addto\captionsrussian{\renewcommand{\contentsname}{Содержание}}

% Формат оглавления
\renewcommand{\cftaftertoctitle}{\thispagestyle{plain}}
\renewcommand\cfttoctitlefont{\hfill\fontsize{16pt}{16pt}\selectfont\bfseries\MakeUppercase}
\renewcommand\cftaftertoctitle{\hfill}

% Добавить точки у разделов в оглавлении
\renewcommand{\cftsecleader}{\cftdotfill{\cftdotsep}} 

% Формат секций и подсекций
\titleformat{\section}{\parskip=6pt\filcenter\fontsize{16pt}{16pt}\selectfont\bfseries\uppercase}{\thesection}{.5em}{}
\titleformat{\subsection}{\bfseries}{\thesubsection}{.5em}{}
\titleformat{\subsubsection}{\bfseries}{\thesubsubsection}{.5em}{}

% Начинать раздел с новой страницы
\AddToHook{cmd/section/before}{\clearpage}

\renewenvironment{abstract}{\clearpage\section*{\MakeUppercase{\abstractname}}}{\clearpage}

% Горизонтальный отступ у элемента списка
\labelwidth=1.25cm 

% Настройка положения нумерации страниц
\fancyhf{}
\renewcommand{\headrulewidth}{0pt}
\pagestyle{fancy}
\fancyhead[R]{\thepage}
\setlength{\headheight}{35pt}

% Ненумерованные списки разной вложенности
\renewcommand\labelitemi{--}
\renewcommand\labelitemii{--}
\renewcommand\labelitemiii{--}
\renewcommand\labelitemiv{--}

% Нумерованные списки разной вложенности
\renewcommand\labelenumi{\arabic{enumi})}
\renewcommand\labelenumii{\asbuk{enumii})}
\renewcommand\labelenumiii{\arabic{enumiii})}
\renewcommand\labelenumiv{\asbuk{enumiv})}

\makeatletter

% Буквы для нумерации списка (исключены ё, з, щ, ч, ъ, ы, ь)
% Подробнее https://ctan.math.illinois.edu/macros/latex/required/babel/contrib/russian/russianb.pdf 
\def\russian@alph#1{\ifcase#1\or
    а\or б\or в\or г\or д\or е\or ж\or
    и\or к\or л\or м\or н\or о\or п\or 
    р\or с\or т\or у\or ф\or х\or ц\or 
    ш\or э\or ю\or я\else\@ctrerr\fi}
\def\russian@Alph#1{\ifcase#1\or
    А\or Б\or В\or Г\or Д\or Е\or Ж\or
    И\or К\or Л\or М\or Н\or О\or П\or 
    Р\or С\or Т\or У\or Ф\or Х\or Ц\or 
    Ш\or Э\or Ю\or Я\else\@ctrerr\fi}

% Очистка страницы с содержанием
\fancypagestyle{plain}{
    % clear all header and footer fields
    \fancyhf{}
    % clear all header fields
    \fancyhead{}
    % clear all footer fields
    \fancyfoot{}
    \fancyhead[RO]{\thepage}
    \fancyhead[LE]{\thepage}
    \setlength{\headheight}{35pt}
    \renewcommand{\headrulewidth}{0pt}
    \renewcommand{\footrulewidth}{0pt}
}

% Разделы в оглавлении без выделения жирным, в верхнем регистре
\patchcmd{\l@section}{#1}{\textnormal{\uppercase{#1}}}{}{}

% Страницы без выделения жирным
\patchcmd{\l@section}{#2}{\textnormal{#2}}{}{}

\apptocmd{\appendix}{
    \renewcommand{\thesection}{\Asbuk{section}}

    % Изменение формата раздела приложения
    \titleformat{\section}{\filcenter\fontsize{16pt}{16pt}\selectfont\bfseries}{}{0pt}{\MakeUppercase{\appendixname}~\thesection \\}{}{}
    
    \let\oldsec\section
    \renewcommand{\section}{
        \clearpage
        \phantomsection
        \refstepcounter{section}
        \addcontentsline{toc}{section}{\appendixname~\thesection}

        % Нумерация раздела после названия
        \oldsec*}
}
\makeatother

% Выравнивание по левому краю надписи таблицы
\captionsetup[table]{justification=raggedright, singlelinecheck=false}

% Переопределение caption из babel
\addto\captionsrussian{\renewcommand{\figurename}{Рисунок}}

% Метаданные .pdf документа, отключение подсветки ссылок
\hypersetup{pdftitle={Исследование методов веторизации текста и извлечения признаков}, pdfauthor={К. А. Проскуряков}, colorlinks=false, pdfborder={0 0 0}}

%%%%%%%%%% Блок описания документа %%%%%%%%%%

\begin{document}
    \begin{titlepage}
        \pagestyle{empty}
        \setlength\parindent{0pt}

        \begin{center}
            Министерство науки и высшего образования Российской Федерации

            ФГАОУ ВО <<Севастопольский государственный университет>>

            Институт информационных технологий
        \end{center}

        \bigskip
        \bigskip

        \begin{flushright}
            Кафедра <<Информационная безопасность>>
        \end{flushright}

        \bigskip
        \bigskip
        \bigskip

        \begin{center}
            \textbf{РАСЧЁТНО-ГРАФИЧЕСКАЯ РАБОТА}

            по теме

            <<Исследование методов векторизации текста

            и извлечения признаков>>

            по дисциплине

            <<Защита программ и данных>>
        \end{center}

        \bigskip
        \bigskip
        \bigskip
        \bigskip

        \begin{flushright}
            \begin{tabularx}{265pt}{lr@{\quad}X}
                \multicolumn{1}{l}{Выполнил:} & \multicolumn{1}{l}{студент гр. ИБ/б-21-1-о} \\
                                              & \multicolumn{1}{l}{Проскуряков К. А.} \\
                                              & \multicolumn{1}{l}{Защитил с оценкой: \rule{1.2cm}{0.25mm}} \\
                \multicolumn{1}{l}{Принял:}   & \multicolumn{1}{l}{доцент Лихолоб П. Г.}
            \end{tabularx}       
        \end{flushright}

        \begin{center}
            \vfill
            Севастополь

            \the\year{}
        \end{center}
    \end{titlepage}

    \addtocounter{page}{1}

    \tableofcontents

    \section*{Введение}
        \phantomsection
        \addcontentsline{toc}{section}{Введение}

        Машинное обучение предоставляет возможность 
        быстро и эффективно решать как внешние, так и внутренние
        задачи, которые возникают перед бизнесом. С каждым днем
        чат-боты становятся всё более совершенными, и отличить
        поведение программы от человеческого становится
        всё сложнее и сложнее.

        С помощью моделей машинного обучения и внедрения
        их в код программы осуществляется генерация уникальных
        ответов в чат-ботах. Для создания эффективной модели
        необходимо осуществить предварительную обработку
        текста, а именно токенизацию, лемматизацию, удаление
        стоп-слов (союзов, предлогов, междометий и т. д.) и
        векторизацию.

        Одним из механизмов классификации текста является
        векторизация. Для обработки используются следующие
        алгоритмы:

        \begin{itemize}
            \item \textit{One-hot encoding};
            \item \textit{TD-IDF};
            \item \textit{CountVectorizer};
            \item \textit{word2vec}.
        \end{itemize}

        Целью работы является проверка алгоритмов векторизации
        на практике и сравнение их эффективности путем
        использования полученного корпуса текстов в методе
        машинного обучения.

        В задачи работы входит:

        \begin{enumerate}
            \item Дать определение используемым в ходе работы
            понятиям;
            \item Применить к полученному корпусу текстов
            алгоритмы векторизации;
            \item Использовать полученный результат в
            качестве моделей для алгоритма машинного обучения;
            \item Сравнить полученный результат.
        \end{enumerate}

        \newpage

        Объектом исследования является текст, предметом --
        методы векторизации текста.

        Работа изложена на INSERT страницах основного текста,
        включающего INSERT рисунков, INSERT таблиц, список
        литературных источников из INSERT наименований,
        INSERT приложений.

    \section{Озаглавить}
        \subsection{Правовое поле использованных библиотек}
            \subsubsection{Библиотеки получения корпуса}
                \begin{itemize}
                    \item \textit{string}
                    -- стандартная библиотека \textit{Python}.
                    Распространяется по лицензии \textit{PSF};
                    \item \textit{re}
                    -- стандартная библиотека \textit{Python}.
                    Распространяется по лицензии \textit{PSF};
                    \item \textit{SpaCy}
                    -- библиотека обработки естественного языка.
                    Распространяется по лицензии \textit{MIT}.
                \end{itemize}

            \subsubsection{Библиотеки векторизации}
                \begin{itemize}
                    \item \textit{sklearn}
                    -- библиотека машинного обучения.
                    Распространяется по лицензии \textit{BSD-3};
                    \item \textit{gensim}
                    -- библиотека обработки естественного
                    языка и информационного поиска.
                    Распространяется по лицензии \textit{LGPL-2.1}.
                \end{itemize}

        \subsection{Глоссарий}
            Текст
            -- это некоторая последовательность предложений,
            имеющая логическую последовательность и сообщающая
            какую-либо информацию.

            Корпус текстов
            -- это подобранная и обработанная по определенным
            правилам совокупность текстов, используемая для
            исследования языка.

            Токен
            -- это текстовая единица (слово, словосочетание
            и т. д.).

        \newpage

        \subsection{Обозначение входных и выходных данных}
            pass

        \newpage

        \subsection{Математические модели методов векторизации}
            \subsubsection{One-hot encoding}
                pass

            \subsubsection{TD-IDF}
                Общая формула показателя $IDF$ выглядит
                следующим образом:

                \begin{equation*}
                    IDF(t, D) = \log{
                        \frac{|D| + 1}{DF(t, D) + 1}
                    }
                \end{equation*}

                где

                \begin{itemize}
                    \item $D$
                    -- количество документов в корпусе,
                    \item $DF(t, D)$
                    -- количество документов, в которых
                    встречается слово.
                \end{itemize}

                Так, если слово встречается во всех документах,
                то $IDF = 0$. В итоге,
                
                \begin{equation*}
                    TFIDF = IDF \cdot TF
                \end{equation*}

            \subsubsection{CountVectorizer}
                pass

            \subsubsection{word2vec}
                pass

    \section{Озаглавить}
        \subsection{Методы предварительной обработки и фильтрации}
            \subsubsection{Токенизация}
                Токенизация представляет из себя процесс
                разбиения больших участков текста на
                абзацы, предложения и слова. Данная операция
                не требует сторонних библиотек и может быть
                реализована с помощью стандартных модулей
                языка \textit{Python}.

                \begin{figure}[H]
                    \centering
                    \includegraphics[width=\textwidth]{images/code-examples/tokenization}
                    \parskip=6pt
                    \caption{Токенизация стандартными библиотеками языка Python}
                    \label{fig:code_tokenization}
                \end{figure}
            
            \subsubsection{Лемматизация}
                Лемматизация -- это процесс приведения слова
                к его словарной (изначальной) форме. Данная
                операция требует подключения сторонних
                библиотек. На рис.~\ref{fig:code_lemmatisation}
                представлен процесс лемматизации с помощью
                библиотеки \textit{SpaCy}.
                
                \begin{figure}[H]
                    \centering
                    \includegraphics[width=\textwidth]{images/code-examples/lemmatisation}
                    \parskip=6pt
                    \caption{Лемматизация библиотекой SpaCy}
                    \label{fig:code_lemmatisation}
                \end{figure}

            \newpage

            \subsubsection{Удаление шумовых слов}
                Под шумовыми словами подразумевают слова, не
                несущие смысловой нагрузки (междометия, союзы
                и т. д.). Операция может быть выполнена средствами
                языка программирования.

                \begin{figure}[H]
                    \centering
                    \includegraphics[width=\textwidth]{images/code-examples/stop-words}
                    \parskip=6pt
                    \caption{Удаление стоп-слов}
                    \label{fig:code_stop-words}
                \end{figure}

    \section{Озаглавить}
        \subsection{}

    \section*{Заключение}

\end{document}
