% % !TeX program = xelatex
\documentclass[14pt, a4paper, titlepage]{extarticle}

%%%%%%%%%% Блок определения подключаемых пакетов %%%%%%%%%%

\usepackage[english,main=russian]{babel}
\usepackage{fontspec}
\usepackage{tabularx}
\usepackage{fancyhdr}
% Поменять гарнитуру в формулах на Times New Roman
\usepackage{newtxmath}
\usepackage[left=30mm, right=15mm, top=20mm, bottom=20mm]{geometry}
% Красная строка у первого абзаца раздела
\usepackage{indentfirst}
\usepackage{graphicx}
% Добавить раздел оглавление в оглавление
\usepackage{tocbibind}
% underline some lines
\usepackage[normalem]{ulem} 
\usepackage{tocloft}
% Настройка раздела, подраздела, подподраздела
\usepackage{titlesec}
% Настройка пунктуации
\usepackage[labelsep=endash]{caption}
% Пакет делает ссылки кликабельными и дает возможность добавить метаданные .pdf документа
\usepackage{hyperref}

%%%%%%%%%% Блок настроек %%%%%%%%%%

% Если возникают проблемы при компиляции с данной строкой, необходимо установить на компьютер Times New Roman
\setmainfont{Times New Roman}

% Размер красной строки
\parindent=1.25cm

% Отступ между абзацами
\parskip=0pt

% Разрешить переносить слоги в 2 буквы (стандартное значение 3)
\righthyphenmin=2

% полуторный межстрочный интервал
\linespread{1.3}

% Настройка заголовка оглавления
\addto\captionsrussian{\renewcommand{\contentsname}{Содержание}}

% Формат оглавления
\renewcommand\cfttoctitlefont{\hfill\fontsize{16pt}{16pt}\selectfont\bfseries\MakeUppercase}
\renewcommand\cftaftertoctitle{\hfill\hfill}

% Добавить точки у разделов в оглавлении
\renewcommand{\cftsecleader}{\cftdotfill{\cftdotsep}} 

% Формат секций и подсекций
\titleformat{\section}{\parskip=6pt\filcenter\fontsize{16pt}{16pt}\selectfont\bfseries\uppercase}{\thesection}{.5em}{}
\titleformat{\subsection}{\bfseries}{\thesubsection}{.5em}{}
\titleformat{\subsubsection}{\bfseries}{\thesubsubsection}{.5em}{}

% Начинать раздел с новой страницы
\AddToHook{cmd/section/before}{\clearpage}

\renewenvironment{abstract}{\clearpage\section*{\MakeUppercase{\abstractname}}}{\clearpage}

% Горизонтальный отступ у элемента списка
\labelwidth=1.25cm 

% Настройка положения нумерации страниц
\fancyhf{}
\renewcommand{\headrulewidth}{0pt}
\pagestyle{fancy}
\fancyhead[R]{\thepage}
\setlength{\headheight}{35pt}

% Ненумерованные списки разной вложенности
\renewcommand\labelitemi{--}
\renewcommand\labelitemii{--}
\renewcommand\labelitemiii{--}
\renewcommand\labelitemiv{--}

% Нумерованные списки разной вложенности
\renewcommand\labelenumi{\arabic{enumi})}
\renewcommand\labelenumii{\asbuk{enumii})}
\renewcommand\labelenumiii{\arabic{enumiii})}
\renewcommand\labelenumiv{\asbuk{enumiv})}

\makeatletter

% Буквы для нумерации списка (исключены ё, з, щ, ч, ъ, ы, ь)
% Подробнее https://ctan.math.illinois.edu/macros/latex/required/babel/contrib/russian/russianb.pdf 
\def\russian@alph#1{\ifcase#1\or
    а\or б\or в\or г\or д\or е\or ж\or
    и\or к\or л\or м\or н\or о\or п\or 
    р\or с\or т\or у\or ф\or х\or ц\or 
    ш\or э\or ю\or я\else\@ctrerr\fi}
\def\russian@Alph#1{\ifcase#1\or
    А\or Б\or В\or Г\or Д\or Е\or Ж\or
    И\or К\or Л\or М\or Н\or О\or П\or 
    Р\or С\or Т\or У\or Ф\or Х\or Ц\or 
    Ш\or Э\or Ю\or Я\else\@ctrerr\fi}

% Очистка страницы с содержанием
\fancypagestyle{plain}{
    % clear all header and footer fields
    \fancyhf{}
    % clear all header fields
    \fancyhead{}
    % clear all footer fields
    \fancyfoot{}
    \fancyhead[RO]{\thepage}
    \fancyhead[LE]{\thepage}
    \setlength{\headheight}{35pt}
    \renewcommand{\headrulewidth}{0pt}
    \renewcommand{\footrulewidth}{0pt}
}

% Разделы в оглавлении без выделения жирным, в верхнем регистре
\patchcmd{\l@section}{#1}{\textnormal{\uppercase{#1}}}{}{}

% Страницы без выделения жирным
\patchcmd{\l@section}{#2}{\textnormal{#2}}{}{}

\apptocmd{\appendix}{
    \renewcommand{\thesection}{\Asbuk{section}}

    % Изменение формата раздела приложения
    \titleformat{\section}{\filcenter\fontsize{16pt}{16pt}\selectfont\bfseries}{}{0pt}{\MakeUppercase{\appendixname}~\thesection \\}{}{}
    
    \let\oldsec\section
    \renewcommand{\section}{
        \clearpage
        \phantomsection
        \refstepcounter{section}
        \addcontentsline{toc}{section}{\appendixname~\thesection}

        % Нумерация раздела после названия
        \oldsec*}
}
\makeatother

% Выравнивание по левому краю надписи таблицы
\captionsetup[table]{justification=raggedright, singlelinecheck=false}

% Переопределение caption из babel
\addto\captionsrussian{\renewcommand{\figurename}{Рисунок}}

% Метаданные .pdf документа, отключение подсветки ссылок
\hypersetup{pdftitle={Исследование методов веторизации текста и извлечения признаков}, pdfauthor={К. А. Проскуряков}, colorlinks=false, pdfborder={0 0 0}}

%%%%%%%%%% Блок описания документа %%%%%%%%%%

\begin{document}
    \begin{titlepage}
        \pagestyle{empty}
        \setlength\parindent{0pt}

        \begin{center}
            Министерство науки и высшего образования Российской Федерации
            \par 
            ФГАОУ ВО <<Севастопольский государственный университет>>
            \par
            Институт информационных технологий
        \end{center}

        \bigskip
        \bigskip

        \begin{flushright}
            Кафедра <<Информационная безопасность>>
        \end{flushright}

        \bigskip
        \bigskip
        \bigskip

        \begin{center}
            \textbf{РАСЧЁТНО-ГРАФИЧЕСКАЯ РАБОТА}
            \par
            по теме
            \par
            <<Исследование методов векторизации текста
            \par
            и извлечения признаков>>
            \par
            по дисциплине
            \par
            <<Защита программ и данных>>
        \end{center}

        \bigskip
        \bigskip
        \bigskip
        \bigskip

        \begin{flushright}
            \begin{tabularx}{265pt}{lr@{\quad}X}
                \multicolumn{1}{l}{Выполнил:} & \multicolumn{1}{l}{студент гр. ИБ/б-21-1-о} \\
                                              & \multicolumn{1}{l}{Проскуряков К. А.} \\
                                              & \multicolumn{1}{l}{Защитил с оценкой: \rule{1.2cm}{0.25mm}} \\
                \multicolumn{1}{l}{Принял:}   & \multicolumn{1}{l}{доцент Лихолоб П. Г.}
            \end{tabularx}       
        \end{flushright}

        \begin{center}
            \vfill
            Севастополь
            \par
            \the\year{}
        \end{center}
    \end{titlepage}

    \addtocounter{page}{1}

    \begin{abstract}
        \subsection*{Метаданные .pdf}
        
        Не забудьте в преамбуле в команде \verb"\hypersetup" поменять значение полей \verb"pdftitle={"Название моего документа\verb"}", \verb"pdfauthor={"Моё имя\verb"}".
    \end{abstract}

    \renewcommand{\cftaftertoctitle}{\thispagestyle{plain}}
    \tableofcontents

    \section*{Введение}
        \phantomsection
        \addcontentsline{toc}{section}{Введение}

        Текст введения.

    \section{Структура документа}
        Структура документа совместима со стандартным классом документа article.

        \subsection{Титульная страница}
            Титульная страница находится в окружении titlepage в начале документа:
            \begin{verbatim}
            \begin{titlepage}
            ...
            \end{titlepage}
            \end{verbatim}

            При изменении кол-ва титульных страниц необходимо отметить кол-во в счетчике станиц. Пример:
            \begin{verbatim}
            \begin{titlepage}
            ...
            \end{titlepage}
            \addtocounter{page}{1}
            \end{verbatim}

        \subsection{Аннотация}
            \begin{verbatim}
            \begin{abstract}
            ...
            \end{abstract}
            \end{verbatim}

        \subsection{Оглавление}
            \begin{verbatim}
            \tableofcontents
            \end{verbatim}

        \subsection{Введение, раздел без нумерации}
            \noindent\verb"\section*{"Введение\verb"}"\\
            \verb"\phantomsection"\\
            \verb"\addcontentsline{toc}{section}{"Введение\verb"}"\\
            \verb"..."

        \subsection{Раздел}
            \begin{verbatim}
            \section{...}
            ...
            \end{verbatim}

        \subsection{Подраздел}
            \begin{verbatim}
            \subsection{...}
            ...
            \end{verbatim}

        \subsection{Пункт}
            \begin{verbatim}
            \subsubsection{...}
            ...
            \end{verbatim}

        \subsubsection{Пример пункта}
            Текст пункта.

        \subsection{Приложение}
            Как и в стандартных классах перед приложениями необходимо указать команду \verb"\appendix".

            Пример с одним приложением:
            \begin{verbatim}
            \appendix
            \section{...}
            ...
            \end{verbatim}

            Пример с тремя приложениями:
            \begin{verbatim}
            \appendix
            \section{...}
            ...
            \section{...}
            ...
            \section{...}
            ...
            \end{verbatim}

            Ссылка на приложение:
            \begin{verbatim}
            \ref{...}
            \end{verbatim}

            Пример ссылки на приложение~\ref{appendix:test_label}.

            За порядком приложений необходимо следить самостоятельно.

        \subsection{Список}
            \begin{verbatim}
            \begin{itemize}
                \item ...,
                ...
            \end{itemize}
            \end{verbatim}

            \begin{itemize}
                \item[] Пример списка:
                \item первый уровень вложенности,
                \begin{itemize}
                    \item второй,
                    \begin{itemize}
                        \item третий,
                        \begin{itemize}
                            \item четвертый.
                        \end{itemize}
                    \end{itemize}
                \end{itemize}
            \end{itemize}

        \subsection{Перечисление}
            \begin{verbatim}
            \begin{enumerate}
                \item ...;
                ...
            \end{enumerate} 
            \end{verbatim}

            \begin{enumerate}
                \item[] Пример перечисления:
                \item первый уровень вложенности;
                \begin{enumerate}
                    \item второй;
                    \begin{enumerate}
                        \item третий;
                        \begin{enumerate}
                            \item четвертый;
                            \item б;
                            \item в;
                            \item г;
                            \item д;
                            \item е;
                            \item ж;
                            \item и;
                            \item к;
                            \item л;
                            \item м;
                            \item н;
                            \item о;
                            \item п;
                            \item р;
                            \item с;
                            \item т;
                            \item у;
                            \item ф;
                            \item x;
                            \item ц;
                            \item ш;
                            \item э;
                            \item ю;
                            \item я, при наличии б\'ольшего кол-ва элементов компилятор выдаст ошибку.
                        \end{enumerate}
                    \end{enumerate}
                \end{enumerate}
            \end{enumerate}

        \subsection{Иллюстрация}
            Пакет graphicx подключен. См. рисунок~\ref{fig:test_label} на стр.~\pageref{fig:test_label}.

            \begin{figure}[htb]
                \centering
                \parskip=6pt
                \caption{Подпись ниже рисунка по центру}
                \label{fig:test_label}
            \end{figure}

            Обратите внимание, что окружение figure является \emph{плавающим}, и иллюстрация может появиться не там, где Вы ожидаете. Для размещения иллюстрации в конкретное место необходимо воспользоваться опцией H из пакета float (не подключен).

        \subsection{Таблица}
            См. таблицу~\ref{tab:test_label} на стр.~\pageref{tab:test_label}.

            \begin{table}[htb]
                \caption{Подпись выше таблицы слева}
                \centering
                \begin{tabular}{ |c|c|c|c|c| } 
                    \hline
                    ячейка 1 & ячейка 2 & ячейка 3 & ячейка 4 & ячейка 5 \\ \hline
                    ячейка 6 & ячейка 7 & ячейка 8 & ячейка 9 & ячейка 10 \\ \hline
                \end{tabular}
                \label{tab:test_label}
            \end{table}

            Обратите внимание, что окружение table является \emph{плавающим}, и таблица может появиться не там, где Вы ожидаете. Для размещения таблицы в конкретное место необходимо воспользоваться опцией H из пакета float (не подключен).

        \subsection{Уравнение и формула}
            См. формулу~\ref{eq:test_label}.

            \begin{equation}\label{eq:test_label}
                \text{минус}\,a\times b=c ,
            \end{equation}
            где $a$ --- первая переменная; \\
            $b$ --- вторая переменная; \\
            $c$ --- третья переменная.

    \appendix
        \section{Использованные пакеты}
            \label{appendix:used_packages}

            \begin{itemize}
                \item caption,
                \item fontspec,
                \item geometry,
                \item graphicx,
                \item hyperref,
                \item indentfirst,
                \item newtxmath,
                \item titlesec,
                \item tocloft,
                \item ulem.
            \end{itemize}

        \section{Название второго приложения}
            \label{appendix:test_label}

            Содержание второго приложения.
\end{document}
